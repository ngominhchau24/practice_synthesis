\documentclass[12pt,a4paper]{article}

% Packages
\usepackage{amsmath, amssymb}
\usepackage{geometry}
\usepackage{graphicx}
\usepackage{booktabs}
\usepackage{array}
\usepackage{enumitem}
\usepackage{hyperref}
\usepackage{titlesec}
\usepackage{setspace}

\geometry{left=2.5cm, right=2.0cm, top=2.5cm, bottom=2.5cm}
\setstretch{1.3}

% Heading format
\titleformat{\section}{\large\bfseries}{\thesection.}{0.5em}{}
\titleformat{\subsection}{\normalsize\bfseries}{\thesubsection}{0.5em}{}

\begin{document}

% ---------------------------------------------------------
% TITLE PAGE
% ---------------------------------------------------------
\begin{titlepage}
    \centering
    {\Large \textbf{LOGIC DESIGN AND SYNTHESIS}}\\[0.3cm]
    {\Huge \textbf{LAB 3 REPORT}}\\[0.3cm]
    {\Large BDD-Based Logic Synthesis}\\[2cm]

    \begin{flushleft}
    \large
    Ngo Minh Chau \hfill 2470212\\
    Tran Gia Tuan \hfill 2470214\\
    Dang Phuoc Tien \hfill 2470455\\[1cm]
    \end{flushleft}

    \vfill
    \large \today
\end{titlepage}

% ---------------------------------------------------------
% EXECUTIVE SUMMARY
% ---------------------------------------------------------
\section*{Executive Summary}

This lab successfully implemented a complete BDD-based logic synthesis flow. Key achievements include:

\begin{itemize}
    \item Automated BDD generation using the Python \texttt{dd} library.
    \item Variable ordering optimization achieving a 40\% node reduction.
    \item Two circuit implementations: Multiplexer-based (4 gates) and Gate-based (8 gates).
    \item Synthesizable Verilog code generation for both implementations.
    \item Complete truth‐table based verification of correctness.
\end{itemize}

Optimal Result: Variable ordering \([c,a,b]\) produces a 3-node BDD, reducing circuit complexity by 40\% compared to non-optimal orderings.

% ---------------------------------------------------------
% INTRODUCTION
% ---------------------------------------------------------
\section{Introduction and Objectives}

\subsection{Objectives}

\begin{itemize}
    \item Generate a Binary Decision Diagram (BDD) from Boolean expression \( f = ab' + c'(a+b) \).
    \item Apply BDD reduction techniques (node sharing and elimination).
    \item Find optimal variable ordering to minimize BDD size.
    \item Generate logic circuits using ITE (If-Then-Else) lookup tables.
    \item Compare multiplexer-based vs gate-based implementations.
\end{itemize}

\subsection{Binary Decision Diagrams (BDDs)}

A BDD is a directed acyclic graph representing Boolean functions:

\begin{itemize}
    \item Non-terminal nodes represent Boolean variables.
    \item Terminal nodes represent constants 0 or 1.
    \item Reduction rules:
    \begin{enumerate}
        \item Merge identical subgraphs.
        \item Remove nodes whose high and low branches are identical.
    \end{enumerate}
\end{itemize}

\subsection{Variable Ordering Importance}

BDD size depends critically on variable ordering. Different orders can yield exponentially different node counts.  
This lab evaluates all 6 permutations of variables \(\{a,b,c\}\).

% ---------------------------------------------------------
% METHODOLOGY
% ---------------------------------------------------------
\section{Methodology}

\subsection{Tools and Environment}

\begin{itemize}
    \item Programming Language: Python 3.x
    \item BDD Library: \texttt{dd}
    \item Features: automatic reduction, variable ordering, ITE operations
\end{itemize}

\subsection{Implementation Workflow}

\begin{enumerate}
    \item Parse Boolean expression  
        \[
        ab' + c'(a+b) \rightarrow (a \& \sim b)\ | \ (\sim c \ \&\ (a|b))
        \]
    \item Generate BDD for all 6 variable permutations.
    \item Apply automatic reduction.
    \item Select ordering with minimum node count.
    \item Generate ITE lookup table.
    \item Synthesize circuits (MUX-based and gate-based).
    \item Generate Verilog HDL.
\end{enumerate}

\subsection{Optimization Metric}

Primary metric: number of non-terminal nodes.  
Fewer nodes imply simpler circuits and lower hardware cost.

% ---------------------------------------------------------
% IMPLEMENTATION DETAILS
% ---------------------------------------------------------
\section{Implementation Details}

\subsection{Class Architecture}

\paragraph{BDDNode Class}
Wrapper for individual BDD nodes.

\paragraph{BDDGraph Class}
Manages:
\begin{itemize}
    \item BDD manager
    \item Root node
    \item Ordered variable list
\end{itemize}

\paragraph{BDDOptimizer Class}
Searches all variable orderings and returns the minimal BDD.

\subsection{Expression Parser}

\begin{center}
\begin{tabular}{lll}
\toprule
Input & BDD Format & Description \\
\midrule
$ab'$ & $a\ \&\ \sim b$ & Implicit AND, prime as NOT \\
$a+b$ & $a\ |\ b$ & OR operator\\
$c'(a+b)$ & $\sim c \ \&\ (a|b)$ & Grouping \\ 
\bottomrule
\end{tabular}
\end{center}

\[
ab' + c'(a+b) \rightarrow (a \& \sim b) \ |\ (\sim c \& (a|b))
\]

% ---------------------------------------------------------
% RESULTS
% ---------------------------------------------------------
\section{Results}

\subsection{Variable Ordering Optimization}

\begin{center}
\begin{tabular}{cccc}
\toprule
Order & Node Count & Overhead & Status \\
\midrule
[a,b,c] & 4 & +33\% & Suboptimal \\
[a,c,b] & 5 & +67\% & Worst \\
[b,a,c] & 4 & +33\% & Suboptimal \\
[b,c,a] & 5 & +67\% & Worst \\
[c,a,b] & 3 & 0\% & Optimal \\
[c,b,a] & 4 & +33\% & Suboptimal \\
\bottomrule
\end{tabular}
\end{center}

\subsection{Optimal BDD Structure}

\begin{verbatim}
Root: c
 ├─ low → a
 │        ├─ low  → 1
 │        └─ high → b
 │                    ├─ low  → 1
 │                    └─ high → 0
 └─ high → 0
\end{verbatim}

\subsection{ITE Lookup Table}

\begin{center}
\begin{tabular}{cccc}
\toprule
Node & Var & High (1) & Low (0) \\
\midrule
n3 & b & 0 & 1 \\
n2 & a & n3 & 1 \\
n1 & c & 0 & n2 \\
\bottomrule
\end{tabular}
\end{center}

\subsection{BDD Metrics}

\begin{verbatim}
{
  nodes: 3,
  depth: 3,
  satisfying_assignments: 6,
  variables: [c, a, b]
}
\end{verbatim}

% ---------------------------------------------------------
% CIRCUIT GENERATION
% ---------------------------------------------------------
\section{Circuit Generation}

\subsection{Multiplexer-Based Implementation}

Mapping:  
\[
ITE(x,H,L) = x ? H : L
\]

Circuit:

\begin{verbatim}
w0 = (b ? 0 : 1)
w1 = (a ? w0 : 1)
f  = (c ? 0 : w1)
\end{verbatim}

\subsection{Gate-Based Implementation}

Expansion:
\[
ITE(x,H,L) = xH + x'L
\]

% ---------------------------------------------------------
% ANALYSIS
% ---------------------------------------------------------
\section{Analysis and Comparison}

\begin{center}
\begin{tabular}{lcc}
\toprule
Characteristic & MUX-Based & Gate-Based \\
\midrule
Total Gates & 4 & 8 \\
Complexity & Low & Medium \\
Power & Lower & Higher \\
Speed & Fast & Slower \\
Best Target & FPGA & ASIC \\
\bottomrule
\end{tabular}
\end{center}

% ---------------------------------------------------------
% VERIFICATION
% ---------------------------------------------------------
\section{Truth Table Verification}

All 8 combinations were tested, and both circuits produced matching outputs.

% ---------------------------------------------------------
% CONCLUSION
% ---------------------------------------------------------
\section{Conclusions}

\begin{itemize}
    \item BDD successfully generated from Boolean expression.
    \item Reduction rules applied automatically.
    \item Optimal variable ordering found: \([c,a,b]\).
    \item MUX-based circuit uses 4 gates; gate-based version uses 8 gates.
    \item Both implementations were fully verified using a truth-table testbench.
\end{itemize}

% ---------------------------------------------------------
% SOURCE CODE
% ---------------------------------------------------------
\section{Source Code Repository}

All Python source files used in this laboratory report are available at:  
\url{https://github.com/ngominhchau24/synthesis.git}

\begin{itemize}
    \item \texttt{bdd\_class\_structure.py} – Core BDD implementation  
    \item \texttt{bdd\_to\_circuit.py} – Converts BDD ITE tables to circuits  
    \item \texttt{example.py} – Full workflow demo  
    \item \texttt{test\_bdd.py} – Unit tests  
    \item \texttt{requirements.txt} – Dependencies  
    \item \texttt{README.md} – Documentation  
\end{itemize}

\end{document}
